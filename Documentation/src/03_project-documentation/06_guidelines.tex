\chapter{Guidelines}
Project guidelines for Kubernetes resources as well as application coding guidelines.

\section{Kubernetes}
\begin{itemize}
    \item Configuration files written in yaml.
    \item Related objects are grouped in a single file if possible.
    \item Resource descriptions in annotations above resource definition.
    \item Use Deployment or StatefulSet to run workloads.
    \item Services are defined before corresponding workload.
    \item Filenames in snake\_case.
    \item Placeholder in angle brackets.
    \item Use label that define semantic attributes of the application or deployment. (Frontend,Backend,etc.)
\end{itemize}

\section{Web-API}
\begin{itemize}
    \item Use WebStorm's default linter to ensure conformity with the basic formatting rules, code smells, syntax, etc.
    \item Use WebStorm's default style guide for TypeScript
    \item Put CSS and JavaScript in separate files
    \item No JavaScript inline styles
    \item No HTML inline styles
    \item Variable names are in camelCase format
    \item All-caps for constant variables
    \item Boolean values start with is..
    \item Arrays have plural names
    \item Function names are verbs written in camelCase format
    \item Functions are short, handle one task, have as few arguments as possible, have a descriptive name, and do nothing else than what is implied
    \item Understandable, easily maintainable code that can be extended in a simple manner with straight forward troubleshooting is important
    \item Use NPM (Node Package Manager) for packages wherever possible
\end{itemize}