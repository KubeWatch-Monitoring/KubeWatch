% File containing the commands and definitions
% that are used throughout this project.

% Set input encoding
\usepackage[utf8]{inputenc}

% For creating coloured text and backgrounds
\usepackage[table,xcdraw]{xcolor}

% For creating compact lists
\usepackage{paralist}

% To get the current date and time
\usepackage{datetime2}

% For including graphics
\usepackage{graphicx}

% For landscape pages
\usepackage{lscape}

% For tables longer than one page (includes page breaks)
\usepackage{longtable}

% Custom commands

\newcommand{\todo}[1]{TODO: {#1}}

% The following command indicates that its content consists of instructions
% Even if it does not do anything, it is still a good idea to keep instructions within the `instruction` tag to separate them from the rest of the text. One could then typeset them differently, or choose to remove them form the document by redefining the command definition 
\newcommand{\instructions}[1]{#1}

% Uncomment the following command to remove all instructions
% \newcommand{\instructions}[1]{}

% Uncomment the following to make instructions appear in coloured boxes
% Note: The changes within `\instruction` are not visible in latexdiff when they are typeset in this way
% \usepackage{tcolorbox}
% \newcommand{\instructions}[1]{
%     \begin{tcolorbox}[sharp corners, colback=green!30, colframe=green!80!blue, title=Instructions]
%         #1
%     \end{tcolorbox}
% }

% Uncomment the following to make instructions appear in colour.
% Note: The changes within `\instructions` are not visible in latexdiff when they are typeset in this way
% \newcommand{\instructions}[1]{ { \color{blue} #1 } }

% Get a git version description roughly using the idea in  
% https://blog.wxm.be/2013/08/20/automated-git-commit-number-in-latex.html
% Note that the file that is included needs to be generated before the document is built.
% Refer to the makefile for further details.
\IfFileExists{../gitDescription.tmp}
{\newcommand{\gitDescription}{\input{../gitDescription.tmp}}}
{\newcommand{\gitDescription}{Not available}}

% Note: Importing hyperref must be done towards the end since it
% redefines many macros
%
% Note:
%  The following packages must be imported *before* hyperref:
%  
%  The following packages must be imported *after* hyperref:
%    amsrefs, geometry
\usepackage{hyperref}

%\usepackage{geometry}
\usepackage[a4paper, left=3cm, right=3cm, top=2cm]{geometry}

\usepackage{listings}

\usepackage{textcomp}

% options for package geometry (influences writable area on page)
% \geometry{a4paper, top=2cm, bottom=2cm}

\usepackage{float}
