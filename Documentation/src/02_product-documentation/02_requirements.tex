\chapter{Requirements}

\section{Functional Requirements}
\label{section:functional-requirements}
The functional requirements are documented as descriptions of user stories, which themself are GitLab issues\footnote{\url{https://gitlab.ost.ch/SEProj/2022-FS/g03-kubewatch/kubewatch/-/issues/55}}.

We use Scrum.
Therefore, all estimates and other refinements are made every two weeks in the sprint meeting.

The User Stories are prioritised.
The higher a User Story is in the Story Board\footnote{\url{https://gitlab.ost.ch/SEProj/2022-FS/g03-kubewatch/kubewatch/-/boards/935}} the higher is the priority.
User stories are ordered into Epics. The more left an Epic is on the Storymap board the higher the priority.

\section{Non-Functional Requirements}
\label{section:non-functional-requirements}
The validate Non-Functional Requirements (NFRs), automation tools are used wherever possible. SonarQube is used as automation tool. Those NFRs which we deem critical, but cannot be tracked by SonarQube will be validated at least once every sprint and confirmed by the validator during a regular sprint meeting. NFRs, which are not met are tracked as technical debt in GitLab issues.

\begin{table}[h!]
    \centering
      \caption{\label{tab:nfr-review}NFR reviews and changelog}
      \begin{tabular}{ | l | l | }
        \hline
        \textbf{Date} & \textbf{Changes} \\
        \hline
        15.03.22 & Created initial set of NFRs. \\
        \hline
        30.03.22 & Added NFR 5-10, Updated NFR-1 \\
        \hline
        01.04.22 & Updated NFR 5,7,8 \\
        \hline
        02.04.22 & Updated NFR 4 \\
        \hline
        22.04.22 & Updated NFR 5, Added NFR 11 \\
        \hline
      \end{tabular}
    \end{table}

\begin{center}
\begin{tabular}{ | m{8em} | m{25em}| } 
 \hline
 \textbf{ID} & NFR-1\\ 
 \hline
 \textbf{Requirement} & The software must have good quality. \\
 \hline
 \textbf{Priority} & High\\
 \hline
 \textbf{Measure(s)} & The test coverage must be at least 75\%.\\
 \hline
 \textbf{Process} & Use \textit{SonarQube} Coverage metrics. \\
 \hline
\end{tabular}
\end{center}

\begin{center}
\begin{tabular}{ | m{8em} | m{25em}| } 
 \hline
 \textbf{ID} & NFR-2\\ 
 \hline
 \textbf{Requirement} & The web interface must load fast.\\
 \hline
 \textbf{Priority} & High\\
 \hline
 \textbf{Measure(s)} & The response time must be under 5s.\\
 \hline
 \textbf{Process} & Use Browsers Developer Tools to verify the loading time.\\
 \hline
\end{tabular}
\end{center}

\begin{center}
\begin{tabular}{ | m{8em} | m{25em}| } 
 \hline
 \textbf{ID} & NFR-3\\ 
 \hline
 \textbf{Requirement} & The application memory usage must be minimal.\\
 \hline
 \textbf{Priority} & Medium\\
 \hline
 \textbf{Measure(s)} & The maximal memory usage must be less than 200 MB.\\
 \hline
 \textbf{Process} & Monitor the application memory usage over an hour and check if memory is not increasing beyond the limit.\\
 \hline
\end{tabular}
\end{center}

\begin{center}
\begin{tabular}{ | m{8em} | m{25em}| } 
 \hline
 \textbf{ID} & NFR-4\\ 
 \hline
 \textbf{Requirement} & The product must be implemented using Node.js and Kubernetes.\\
 \hline
 \textbf{Priority} & Low\\
 \hline
 \textbf{Measure(s)} & Only JS, TS (Application) and YAML (Kubernetes) files are used.\\
 \hline
 \textbf{Process} & Check the source code and the deployment infrastructure.\\
 \hline
\end{tabular}
\end{center}

\begin{center}
\begin{tabular}{ | m{8em} | m{25em}| } 
 \hline
 \textbf{ID} & NFR-5\\ 
 \hline
 \textbf{Requirement} & The product must run reliable.\\
 \hline
 \textbf{Priority} & Middle\\
 \hline
 \textbf{Measure(s)} & Software must run in a Kubernetes pod.\\
 \hline
 \textbf{Process} & Let the software run in a Kubernetes deployment.\\
 \hline
\end{tabular}
\end{center}

\begin{center}
\begin{tabular}{ | m{8em} | m{25em}| } 
 \hline
 \textbf{ID} & NFR-6\\ 
 \hline
 \textbf{Requirement} & The Software must usable for a technical user.\\
 \hline
 \textbf{Priority} & High \\
 \hline
 \textbf{Measure(s)} & Technical User gives positive feedback.\\
 \hline
 \textbf{Process} & Have a technical user work with the software.\\
 \hline
\end{tabular}
\end{center}

\begin{center}
\begin{tabular}{ | m{8em} | m{25em}| } 
 \hline
 \textbf{ID} & NFR-7\\ 
 \hline
 \textbf{Requirement} & The software must handle absence of APIs it depends on.\\
 \hline
 \textbf{Priority} & High \\
 \hline
 \textbf{Measure(s)} & Software sends notifications about missing APIs.\\
 \hline
 \textbf{Process} & Let the software run without connection to any API.\\
 \hline
\end{tabular}
\end{center}

\begin{center}
\begin{tabular}{ | m{8em} | m{25em}| } 
 \hline
 \textbf{ID} & NFR-8\\ 
 \hline
 \textbf{Requirement} & The software must reconnect to APIs it depends on automatically.\\
 \hline
 \textbf{Priority} & Middle \\
 \hline
 \textbf{Measure(s)} & The software sends a Notification that it is fully functional again.\\
 \hline
 \textbf{Process} & Start software without connection to any API and start APIs after software.\\
 \hline
\end{tabular}
\end{center}

\begin{center}
\begin{tabular}{ | m{8em} | m{25em}| } 
 \hline
 \textbf{ID} & NFR-9\\ 
 \hline
 \textbf{Requirement} & The software must be secure. \\
 \hline
 \textbf{Priority} & Middle \\
 \hline
 \textbf{Measure(s)} & The Security Rating must be at least a B. \\
 \hline
 \textbf{Process} & Use \textit{SonarQube} Security Rating metrics. \\
 \hline
\end{tabular}
\end{center}

\begin{center}
\begin{tabular}{ | m{8em} | m{25em}| } 
 \hline
 \textbf{ID} & NFR-10\\ 
 \hline
 \textbf{Requirement} & Code must be maintainable. \\
 \hline
 \textbf{Priority} & High \\
 \hline
 \textbf{Measure(s)} & The Maintainability Rating must be at least a B.\\
 \hline
 \textbf{Process} & Use \textit{SonarQube} Maintainability Rating metrics. \\
 \hline
\end{tabular}
\end{center}


\begin{center}
\begin{tabular}{ | m{8em} | m{25em}| } 
 \hline
 \textbf{ID} & NFR-11\\ 
 \hline
 \textbf{Requirement} & The product scales to show as many Kubernetes instances of the monitored Kubernetes resources as Prometheus. \\
 \hline
 \textbf{Priority} & Middle\\
 \hline
 \textbf{Measure(s)} & Software is able to list 10 Pods. \\
 \hline
 \textbf{Process} & Let the software run with at least 10 pods.\\
 \hline
\end{tabular}
\end{center}
