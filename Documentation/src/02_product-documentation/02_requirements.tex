\chapter{Requirements}

% \instructions{
%     Describe the functional and non-functional requirements as covered in the SEP1 and SEP2 modules.
% }


\section{Functional Requirements}
The functional requirements are written as a description in the GitLab issue\footnote{\url{https://gitlab.ost.ch/SEProj/2022-FS/g03-kubewatch/kubewatch/-/issues/55}}.

We do use Scrum.
Therefore, all estimates and other refinements are made every two weeks in the sprint meeting.

The User Stories are prioritised.
The higher a User Story is in the Story Board\footnote{\url{https://gitlab.ost.ch/SEProj/2022-FS/g03-kubewatch/kubewatch/-/boards/935}} the higher is the priority.



\section{Non-Functional Requirements}
\subsection{Test Coverage}
\begin{description}
\item[Criteria] The test coverage must be minimal 75\%.
\item[Priority] Medium
\item[Check] Use the test coverage tool \textsl{Istanbul}.
\end{description}

\subsection{Response Time}
\begin{description}
\item[Criteria] The response time must be under 2s.
\item[Priority] High
\item[Check] Use Browsers Developer Tools to verify the loading time.
\end{description}

\subsection{Memory Usage}
\begin{description}
\item[Criteria] The maximal memory usage must be less than 200 MB.
\item[Priority] Medium
\item[Check] Monitor the system over a given period of time.
\end{description}

\subsection{Scalability}
\begin{description}
\item[Criteria] The product must support minimal 3 pods.
\item[Priority] High
\item[Check] Install 3 pods and use it in the Frontend.
\end{description}

\subsection{Environment}
\begin{description}
\item[Criteria] The product must be implemented using Node.js and Kubernetes.
\item[Priority] Low
\item[Check] Check the source code and the deployment infrastructure.
\end{description}