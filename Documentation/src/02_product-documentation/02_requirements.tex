\chapter{Requirements}

% \instructions{
%     Describe the functional and non-functional requirements as covered in the SEP1 and SEP2 modules.
% }


\section{Functional Requirements}
The functional requirements are written as a description in the GitLab issue\footnote{\url{https://gitlab.ost.ch/SEProj/2022-FS/g03-kubewatch/kubewatch/-/issues/55}}.

We do use Scrum.
Therefore, all estimates and other refinements are made every two weeks in the sprint meeting.

The User Stories are prioritised.
The higher a User Story is in the Story Board\footnote{\url{https://gitlab.ost.ch/SEProj/2022-FS/g03-kubewatch/kubewatch/-/boards/935}} the higher is the priority.


\section{Non-Functional Requirements}
The process for validating Non-Functional Requirements (NFRs) is to use automation tools wherever possible. We are currently setting up SonarQube and we will add as many of the defined metrics as possible. Those NFRs which we deem critical, but cannot be tracked by SonarQube will be validated at least once every sprint and confirmed by the validator during a regular sprint meeting.

\begin{table}[h!]
    \centering
      \caption{\label{tab:nfr-review}NFR reviews and changelog}
      \begin{tabular}{ | l | l | }
        \hline
        \textbf{Date} & \textbf{Changes} \\
        \hline
        15.03.22 & Created initial set of NFRs. \\ \hline
      \end{tabular}
    \end{table}

\begin{center}
\begin{tabular}{ | m{8em} | m{25em}| } 
 \hline
 \textbf{ID} & NFR-1\\ 
 \hline
 \textbf{Requirement} & Software must be tested\\  
 \hline
 \textbf{Priority} & Medium\\
 \hline
 \textbf{Measure(s)} & The test coverage must be minimal 75\%\\
 \hline
 \textbf{Process} & Metrics from \textsl{SonarQube}\\
 \hline
\end{tabular}
\end{center}

\begin{center}
\begin{tabular}{ | m{8em} | m{25em}| } 
 \hline
 \textbf{ID} & NFR-2\\ 
 \hline
 \textbf{Requirement} & The web interface must be loaded fast\\
 \hline
 \textbf{Priority} & High\\
 \hline
 \textbf{Measure(s)} & The response time must be under 5s\\
 \hline
 \textbf{Process} & Use Browsers Developer Tools to verify the loading time\\
 \hline
\end{tabular}
\end{center}

\begin{center}
\begin{tabular}{ | m{8em} | m{25em}| } 
 \hline
 \textbf{ID} & NFR-3\\ 
 \hline
 \textbf{Requirement} & The application memory usage must be minimal\\
 \hline
 \textbf{Priority} & Medium\\
 \hline
 \textbf{Measure(s)} & The maximal memory usage must be less than 200 MB\\
 \hline
 \textbf{Process} & Monitor the system over a given period of time\\
 \hline
\end{tabular}
\end{center}

\begin{center}
\begin{tabular}{ | m{8em} | m{25em}| } 
 \hline
 \textbf{ID} & NFR-4\\ 
 \hline
 \textbf{Requirement} & The product must be implemented using Node.js and Kubernetes\\
 \hline
 \textbf{Priority} & Low\\
 \hline
 \textbf{Measure(s)} & Only JS / TS files are used\\
 \hline
 \textbf{Process} & Check the source code and the deployment infrastructure\\
 \hline
\end{tabular}
\end{center}


