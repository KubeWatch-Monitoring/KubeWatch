\chapter{Test Concept}

\section{Test Concept - Software}
To develop software with good quality we need to test it.
For this project three different types of tests are used:

\begin{itemize}
  \item Unit Tests
  \item Integration Tests
  \item Usability Tests
\end{itemize}

\subsection{Unit Test}
The Unit Tests are done using the \textit{mocha} and \textit{chai} framework.

\subsection{Integration Test}
The Integration Tests are also done using the \textit{mocha} and \textit{chai} framework.
But \textit{chai} is extended using the \textit{chai-http} and \textit{chai-dom} framework.
Using \textit{chai-http} HTTP request are simulated and \textit{chai-dom} is used to make assertion about the delivered HTML page.

\subsection{Usability Test}
To test the usability of the application we demonstrate and interact with Laurent Metzger, the advisor.
Because he is a technical user he is perfectly suited to judge the application for its usability.

\subsection{Integration into Workflow}
Before pushing any changes you must run all Unit and Integrations tests.
If all tests are passed you can push the changes.
On the server side, we use the GitLab CI/CD pipeline to test with the same tests.
Only if they pass, the merge request can be merged into the master branch.

To run all tests in the IDE a run configuration is provided.


\section{Test Concept - NFR}
Therefore we need to test also the non-functional requirements in our project. Some of the NFRs can be tested by SonarQube like the test coverage (NFR-1) the software security (NFR-9) and the code maintainability (NFR-10).

For the other non-functional requirements we need a concept to verify that the requirements be checked regularly. Like updating the threat model and the risk management it makes sense to include this process directly in each sprint.

Each NFR describe the test process for the verification and is placed directly by the NFR. So once per sprint one team member needs to check all these NFRs excluding these which are checked by \textit{Sonar Qube}.

\section{Test Concept - Security}
To complete the test concept a section for security testing is needed. For this test concept, the \href{OWASP Top Ten}{https://owasp.org/www-project-top-ten/} are used as a guideline and build the basics of the security test concept.
Following all the top ten security risks are listed and on each part, there is a description of how we test it.

\subsection{A01 - Brocken Access Control}
After implementing the login it's important to test our application for broken access control.

\begin{itemize}
  \item Principle of least privilege
  \item IDOR Attack (modifying URL to get access without login)
  \item Login without an account (NoSQL Injection)
\end{itemize}

\subsection{A02 - Cryptographic Failures}
???

\subsection{A03 - Injection}
Another important security risk is the injection. The list below includes some of the most injection attacks which should be avoided.
\begin{itemize}
  %\item SQL Injection - gibt es in dieser Arbeit nicht
  \item NoSQL Injection
  \item XSS
  \item CSRF
  \item ORM (object-relational map)
\end{itemize}

% ev. in gitlab ci/cd einbinden, es gibt anscheinend tools dafür

\subsection{A04 - Insecure Design}
To avoid this security risk we use the \textit{Sonar Qube} to test libraries and implementations for bugs. In addition to this mitigation, we use threat modelling to analyse our application every second week and try to find mitigation techniques to prevent the risks, which is also called a secure development lifecycle.

\subsection{A05 - Security Misconfiguration}
This security risk includes unnecessary features, incorrect error handling, default passwords, etc.

To prevent this risk we just use minimal platforms without unnecessary features and review each implementation to check twice the usage of the feature and the correct implementation.

\subsection{A06 - Vulnerable and Outdated Components}
In this section, the security risk discusses the usage of outdated and vulnerable components. Therefore not only the directly used but also the third-party components are relevant to monitor. For this check, we use the \textit{SonarQube} tool which also checks the test coverage of our code. This tool searches for bugs and vulnerable libraries which are used by our code.

\subsection{A07 - Identification and Authentication Failures}
This category defined a correct authentication but with failures. This can be allowing default, weak, or well-known passwords, allowing brute force, bad hashing algorithms etc.

This security risk isn't tested now in our application. The reason for this is, that we didn't implement an authentication til now so we can't test this future risk.

\subsection{A08 - Software and Data Integrity Failures}
This risk describes insecure data integrity by software or CI/CD pipelines.

To prevent this risk we use npm for libraries installation because it's a trusted repository. Therefore we review each process that is newly implemented or changed, this helps us to find such integrity failures.

\subsection{A09 - Security Logging and Monitoring Failures}
In this security risk, the logging and monitoring failures are described. This can be no logging, no alerts, unclear log messages etc.

For us, this isn't that relevant until now because we didn't implement logging but when we implement the logging we need to monitor each login try (failed logins and successful logins). It's important to check the monitoring and logging cycle during the review of the code implementation and discuss this topic after the implementation in the next team meeting.

\subsection{A10 - Server-Side Request Forgery}
???