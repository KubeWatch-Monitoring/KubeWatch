\chapter{Quality Measures}

%% \instructions{
%%     Describe the quality measures applied in your project as covered in the SEP2 module. Things that might be included in this chapter:
    
%%     \begin{itemize}
%%         \item Organizational means like \textit{Merge Requests}, \textit{Definition of Done}, etc.
%%         \item Tools used to assess the quality of your product (linter, metrics, ...)
%%         \item Tools used to build and deploy your product (CI/CD)
%%         \item The \textit{Test Concept} used for testing your product
%%     \end{itemize}
    
%%     Try to avoid duplication with other chapters such as the \textit{Project Plan}. Work with cross-references when appropriate.
%% }

\section{Organizational}

\subsection{Definition of Ready - DoR}

\begin{itemize}
  \item The item has a clear business value
  \item Dependencies are identified, and no external dependencies will block the item
  \item The team is staffed appropriately to complete the item
  \item Acceptance criteria are defined
    \begin{itemize}
      \item Functional Requirements (see section \ref{functional-requirements})
      \item Non-Functional Requirements (see section \ref{functional-requirements})
    \end{itemize}
\end{itemize}

\subsection{Definition of Done - DoD}
The following \textit{Definition of Done} was taken and adjusted to our needs from scrum-events.de \cite{www.scrum-events.de_dod}.
\begin{itemize}
  \item The code is finished and versioned by git
  \item The code was review or developed using pair programming
  \item The code fulfills the coding guidelines in \ref{guidelines-web-api}
  \item The code has unit tests (where possible) and these are green
  \item The documentation was updated
  \item No critical bugs are known
\end{itemize}

\section{Apply Quality standard}
To provide a certain code standard and quality coding guidelines were specified.
These guidelines can be found in the section \ref{guidelines-web-api}.
For the Kubernetes similar guidelines were specified.
You can find them in section \ref{guidelines-kubernetes}.

In addition, the changes must be reviewed by another team member.
For more information, see the section \ref{git-workflow}.

To ensure that the code base has a certain quality \textit{SonarQube} is used.
Installation see chapter \ref{sonarqube}.
Quality measures are defined in the section \ref{non-functional-requirements} as Non-Functional Requirements.


\section{Test Concept - Software}
To develop software with good quality we need to test it.
For this project three different types of tests are used:

\begin{itemize}
  \item Unit Tests
  \item Integration Tests
  \item Usability Tests
\end{itemize}

\subsection{Unit Test}
The Unit Tests are done using the \textit{mocha} and \textit{chai} framework.

\subsection{Integration Test}
The Integration Tests are also done using the \textit{mocha} and \textit{chai} framework.
But \textit{chai} is extended using the \textit{chai-http} and \textit{chai-dom} framework.
Using \textit{chai-http} HTTP request are simulated and \textit{chai-dom} is used to make assertion about the delivered HTML page.

\subsection{Usability Test}
To test the usability of the application we demonstrate and interact with Laurent Metzger, the advisor.
Because he is a technical user he is perfectly suited to judge the application for its usability.

\subsection{Integration into Workflow}
Before pushing any changes you must run all Unit and Integrations Test.
If all tests are passed you can push the changes.
On the server side we use the GitLab CI/CD pipeline to test with the same tests.
Only if they pass, merge request can be merged into the master branch.

To run all tests in the IDE a run configuration is provided.


\section{Test Concept - NFR}
Therefore we need to test also the non-functional requirements in our project. Some of the NFRs can be tested by SonarQube like the test coverage (NFR-1) the software security (NFR-9) and the code maintainability (NFR-10).

For the other non-functional requirements we need a concept to verify that the requirements be checked regularly. Like updating the threat model and the risk management it make sense to include this process directly in each sprint.

Each NFR describe the test process for the verification and is placed directly by the NFR. So once per sprint one team-member needs to check all these NFRs excluding these which are checked by \textit{Sonar Qube}.

\section{Test Concept - Security}
To complete the test concept a section for security testing is needed. For this test concept the \href{OWASP Top Ten}{https://owasp.org/www-project-top-ten/} are used as guidline and build the basics of the security test concept.
Following all the top ten security risks are listed and on each part there is a description how we test it.

\subsection{A01 - Brocken Access Control}
After implementing the login it's important to test our application for brocken access control.

\begin{itemize}
  \item Principle of least privilege
  \item IDOR Attack (modifying URL to get access without login)
  \item Login without an account (NoSQL Injection)
\end{itemize}

\subsection{A02 - Cryptographic Failures}
???

\subsection{A03 - Injection}


\subsection{A04 - Insecure Design}
\subsection{A05 - Security Misconfiguration}
\subsection{A06 - Vulnerable and Outdated Components}
\subsection{A07 - Identification and Authentication Failures}
\subsection{A08 - Software and Data Integrity Failures}
\subsection{A09 - Security Logging and Monitoring Failures}
\subsection{A10 - Server-Side Request Forgery}