\chapter{Quality Measures}

%% \instructions{
%%     Describe the quality measures applied in your project as covered in the SEP2 module. Things that might be included in this chapter:
    
%%     \begin{itemize}
%%         \item Organizational means like \textit{Merge Requests}, \textit{Definition of Done}, etc.
%%         \item Tools used to assess the quality of your product (linter, metrics, ...)
%%         \item Tools used to build and deploy your product (CI/CD)
%%         \item The \textit{Test Concept} used for testing your product
%%     \end{itemize}
    
%%     Try to avoid duplication with other chapters such as the \textit{Project Plan}. Work with cross-references when appropriate.
%% }

\section{Organizational}

\subsection{Definition of Ready - DoR}

\begin{itemize}
  \item The item has a clear business value
  \item Dependencies are identified, and no external dependencies will block the item
  \item The team is staffed appropriately to complete the item
  \item Acceptance criteria are defined
    \begin{itemize}
      \item Functional Requirements (see section \ref{functional-requirements})
      \item Non-Functional Requirements (see section \ref{functional-requirements})
    \end{itemize}
\end{itemize}

\subsection{Definition of Done - DoD}
The following \textit{Definition of Done} was taken and adjusted to our needs from scrum-events.de \cite{www.scrum-events.de_dod}.
\begin{itemize}
  \item The code is finished and versioned by git
  \item The code was a review or developed using pair programming
  \item The code fulfils the coding guidelines from section \ref{guidelines-web-application}
  \item The code has unit tests (where possible) and these are green
  \item The documentation was updated
  \item No critical bugs are known
\end{itemize}

\section{Apply Quality Standard}
To provide a certain code standard and quality coding guidelines were specified.
These guidelines can be found in section \ref{guidelines-web-application}.
For the Kubernetes similar guidelines were specified.
You can find them in section \ref{guidelines-kubernetes}.

In addition, changes to the code must be reviewed by another team member.
For more information, see section \ref{git-workflow}.

To ensure that the code base has a certain quality \textit{SonarQube} is used because this tool checks the test coverage and in the same step it tests the software for bugs, which is important for us.
Installation see chapter \ref{sonarqube}.
Quality measures are defined in section \ref{non-functional-requirements} as Non-Functional Requirements.