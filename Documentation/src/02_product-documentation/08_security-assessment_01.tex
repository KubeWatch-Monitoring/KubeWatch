\chapter{Security Assessment 21.04.2022}

Based on the security test concept and the potential threats which we identified as part of the Threat Model, we performed a brief assessment of how secure the KubeWatch application is currently.

\section{Scope}

Most of the OWASP top 10 security risks do not apply to the KubeWatch application since no user login is implemented. However, there are still a few that remain and should be tested.

The remaining risks to test are: Injection and Server-Side Request Forgery. 

\subsection{Injection attack}
\subsubsection{Brief description}
Injection attacks can occur on two levels in the KubeWatch application: on the KubeWatch Backend API and Prometheus. However, the second is not relevant for this assessment since the only queries are hard-coded and submitted by the backend only.

The KubeWatch Backend API is connected to a MongoDB database which is NoSQL-based. NoSQL injection attacks are possible by using the JSON structure to send queries to the database which are processed as valid commands.

\subsubsection{Attack}
For our assessment, we tried to query the database so that it would return the top-level entry. This can be done by sending the following string via an input field that is supposed to query the database: \lstinline "'{$ne: null}'".

There are currently two fields which take inputs and store them in the MongoDB database. It is on the \textit{/users} page and its current purpose is to demonstrate that one could create new entries in the database. Later, this should allow to create new users, but this is not implemented yet and is out of scope for this assessment.

\subsubsection{Result}
The attack was unsuccessful. This is because of three types of input validation on two different layers.

Firstly, the input forms in the frontend are of type text and type email. This achieves that any input to both fields is treated as text, and additionally, type email requires the specific email format so that no other sequence of characters is allowed. 

Secondly, handlebars are used on the backend which have the benefit when using two curly brackets to retrieve values from a variable, e.g. \lstinline "{{data}}", any value is interpreted as a string. However, if three curly brackets were used, the data input is treated as an object, which would allow execution of e.g. HTML syntax.

Thirdly, since Typescript is used on the backend, the class which is used to store user inputs contains two variables for each input field. Both required the input to be of type string.

These three defence mechanisms thwart any type of injection attack.

\subsection{Server-Side Request Forgery}
\subsubsection{Brief description}
SSRF can currently be attempted on the \textit{/notifications} page, which is temporarily set up to allow the testing of notifications. This page allows two user inputs: one to trigger a test notification, and the second provides a reason to silence the notification.

The triggering of notifications will be disabled in the future, however, the silencing input will remain to allow a user to close any notifications. Since the silencing reason will be stored in the database, there may be injection attacks or even SSRF attacks possible.

\subsubsection{Attack}
Similar to the injection attack, we tried to first trigger the pop-up of an alert using a script: \lstinline "<script>alert('42')</script>". Also, we tried to call a common localhost address, e.g. \lstinline "http://localhost:8080". 

\subsubsection{Result}
Both attempts were unsuccessful. This is again because of input validation on the front- and backend as seen previously in the injection attack.

The frontend input fields allow text only, and the rendering is again performed by handlebars. Both elements are correctly implemented using \lstinline "type=text" and using double curly brackets for the handlebars.

Also, since Typescript is used, the class for handling notifications requires the notification reason and the silencing text to be stored as a string.

All three elements do prevent any successful attempt to either perform an injection or SSRF attack.


\subsection{Assessment}
As of commit \lstinline "f144aba6" from 21 April 2022, we did not find any known vulnerabilities in the KubeWatch application by testing based on the security test concept.
