\chapter{Vision}

%\textbf{KubeWatch} is a monitoring web application to monitor Kubernetes clusters. This application is designed for technical users and requires some basic knowledge about Kubernetes. With KubeWatch you can visualise and summarise various metrics of a cluster or individual nodes and different Kubernetes workloads. With an adaptive interface, each user can view the metrics and visualisations according to their own preferences. In addition to visualisation, KubeWatch has an alert function, for which thresholds on different metrics can be set in the web application by individual users. KubeWatch will use these metrics to calculate whether an alert needs to be sent or whether the value is still within the desired range. The notifications are shown/sent as simple pop up messages, via email or other messaging applications. With this feature set, KubeWatch allows for easy monitoring and user-defined alerting.

\textbf{KubeWatch} is a monitoring application designed for Kubernetes.
It aims to help Kubernetes administrators and users to monitor and visualize their Kubernetes cluster.
The goal of our application is to provide a simple yet useful interface to manage your clusters.
Especially now, as Kubernetes clusters are becoming more and more important and widespread,
we want to provide a user-friendly application to conveniently monitor these clusters.
KubeWatch is designed for technical users who have at least basic knowledge of Kubernetes.
Nevertheless, the application should not be too complex, so that everyone can quickly find their way around in it.

\noindent
In comparison to the already existing Kubernetes monitoring applications, KubeWatch will serve as a useful extension.
Though the visualization of the monitored cluster in a graph will definitely be a unique selling proposition.
To our knowledge, such a feature does not yet exist on the market.
The visualization will be the big challenge of this project but the benefits of such a useful feature outweigh the risks in our opinion.
Especially in virtual environments it is often difficult to get an overview of the relationships between the components,
which we aim to change with KubeWatch.
Additionally, the application will include a notification infrastructure that supports multiple communication channels.
It will notify the responsible people if anything does not function as usual or if a problem occurred.
