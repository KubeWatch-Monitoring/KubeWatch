\chapter{Vision}

%\textbf{KubeWatch} is a monitoring web application to monitor Kubernetes clusters. This application is designed for technical users and requires some basic knowledge about Kubernetes. With KubeWatch you can visualise and summarise various metrics of a cluster or individual nodes and different Kubernetes workloads. With an adaptive interface, each user can view the metrics and visualisations according to their own preferences. In addition to visualisation, KubeWatch has an alert function, for which thresholds on different metrics can be set in the web application by individual users. KubeWatch will use these metrics to calculate whether an alert needs to be sent or whether the value is still within the desired range. The notifications are shown/sent as simple pop up messages, via email or other messaging applications. With this feature set, KubeWatch allows for easy monitoring and user-defined alerting.

\textbf{KubeWatch} is a monitoring application designed for Kubernetes clusters. With the help of this application it will be easier for Kubernetes users to monitor and visualize their cluster. The goal of our KubeWatch application is to provide a simple yet complete interface to manage your clusters. Especially at this time when Kubernetes clusters are becoming more and more important and widespread, it is necessary to provide a user-friendly application for them. KubeWatch is designed for technical users who have a basic knowledge of Kubernetes. Nevertheless, the application should not be too complex, so that everyone can quickly find the way around. 
In contrast to the already existing Kubernetes monitoring applications, KubeWatch should serve as an extension. With a built-in visualization of the monitoring cluster we would like to achieve a peculiarity. We have not yet discovered such a visualization on the market. The visualization will be the big challenge of this application but the benefit we get from it is even bigger. Especially in virtual environments it is often difficult for us to get an idea of the interrelationships, but this should be changed by KubeWatch. In addition to the visualization, we will also set up a notification. It is important to us that users are notified directly if their cluster does not function as usual and a problem occurs somewhere.