\chapter{Usability Test 23.05.2022}
To test our application we planed for the project a usability test. This test will help us to check the application of intuitivity and self explanation.

The usability test was performed by Marco Agostini and the results are documented in this chapter.

\section{Results}
\subsection{Dashboard}
\begin{itemize}
    \item In the charts there is actualy a button which is called "Delete". To make it clearlier which action is placed behind this button we should change the button name to "Delete Chart".
    \item It is confusing that the charts are shown in "Bytes" because it does not make a lot of sense to show these data in bytes.
    \item In each chart ther is another button callen "Collapse" but when you click the button and the chart is collapsed the name of the button should change to "Expand".
    \item It will be a lot more manageable when there is another button on each chart to change the chart directly without a new creation of the chart.
    \item There should be auto scaling for the values in the charts.
\end{itemize}

\subsection{List all Pods}
\begin{itemize}
    \item When the database lost the connection there should somewhere be a notification for, actually only a button appear but it will be clearer if there is also a text (maybe in red).
    \item Often this view shows database unavailable (why?).
    \item In the Pod descripter there is \textit{Health} and \textit{State} mixed together, because "Running" is a \textit{State} the \textit{Health}.
\end{itemize}

\subsection{Cluster Visualization}
\begin{itemize}
    \item Add a cluster vis zoom into the objects.
    \item For better visualization use different colors for different things.
    \item It will be a lot more handable if you can click in the cluster visualization to get the pod description directly.
\end{itemize}

\subsection{Notification Overview}
\begin{itemize}
    \item At the bottom of the notification table there is a line which is a little bit confusing because at the top of the table is not a line too.
\end{itemize}

\subsection{Settings}
\begin{itemize}
    \item On this side there is often the world "Settings". Maybe we can change the name a little bit, but it will not confuse that much.
\end{itemize}

\subsection{Edit Dashboard}
\begin{itemize}
    \item The oldest and newest datapoint description is not that clear, so a user will not understand what he needs to enter in this two fields.
    \item For better usability it will make sense to integreate the \textit{Edit Dashboard} page in the \textit{Dashboard} because when you like to edit a chart you search on the chart page (Dashboard) and not in the navigation bar for this option.
\end{itemize}

\subsection{General}
\begin{itemize}
    \item Make a clear navigation bar to separate them a little bit form the content of the page.
    \item The page headers are not always in the same size.
\end{itemize}